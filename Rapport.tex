%%
%
% Gnikwo
% gnikwo@hotmail.com
%
%%

\documentclass[a4paper]{report}

%packages
\usepackage[utf8]{inputenc}
\usepackage[francais]{babel}
\usepackage{graphicx}\graphicspath{{pix/}}
\usepackage{float}
\usepackage[T1]{fontenc}
\usepackage{color}
\usepackage{fancyhdr}
%Options: Sonny, Lenny, Glenn, Conny, Rejne, Bjarne, Bjornstrup
\usepackage[Bjornstrup]{fncychap}
\usepackage[procnames]{listings}
\usepackage[colorlinks=true,linkcolor=black]{hyperref}
\usepackage{pdfpages}
\usepackage{titlesec, blindtext, color}

%pdf metadata
\hypersetup{
    unicode=true,
    colorlinks=true,
    citecolor=black,
    filecolor=black,
    linkcolor=black,
    urlcolor=black,
    pdfauthor={Nicolas Ballet <gnikwo@hotmail.com>},
    pdftitle={},
    pdfcreator={pdftex},
    pdfsubject={},
    pdfkeywords={},
}


\definecolor{keywords}{RGB}{200,0,90}
\definecolor{comments}{RGB}{0,0,113}
\definecolor{red}{RGB}{160,0,0}
\definecolor{green}{RGB}{0,150,0}

%inner meta
\title{Rapport LO43}
\author{Nicolas Ballet, Nicolas Vincent, Marine Collet, Lucas Demouy}
\date{\today}


\begin{document}

\maketitle

\tableofcontents
\listoffigures

\chapter{Background}

\section{Attentes}

Le but de ce projet était de montrer que nous avons appris à créer un programme ayant une structure générique et
réutilisable dans le cadre de l'UV LO43.

\section{Choix du sujet}

Nous étions interessés par l'un des sujets proposés par M.XXX: "Développement d'un framework permettant de créer et de
gèrer des interfaces graphiques". Initialement, ce sujet propose donc de concevoir et de développer un framework en
Java permettant, lorsque nous avons une application déja éxistante de pouvoir y greffer une interface graphique, et cela
de manière simple. On doit donc pouvoir définir des paramètres liés à l'application, ainsi que des boutons contrôlant
son fil d'éxécution (Start / Stop / Pause / Pas à pas).
S'approprier son sujet étant très important, nous y avons apporté des
modifications. Nous sommes donc partis sur un développement en C++, utilisant OpenGL pour le rendu graphique, et
utilisant des threads afin de séparer l'interface graphique et le coeur de l'application. Le
framework doit pouvoir etre compilé aussi bien sur Windows que sur un système Unix.

Nos modifications impliquaient l'implémentation d'un moteur graphique. Mais cela faisait aussi appel à beaucoup de
générissité, nous nous sommes donc concentrés la dessus.

\section{Objectifs}

Finalement, notre sujet s'est formalisé comme ceci:
Création d'un framework multi-os performant, permettant de créer des interfaces sur des applications pré-éxistantes ou
non. Il devra être capable d'être étendu simplement (l'utilisateur du framework doit pouvoir facilement créer de
nouveaux types d'éléments graphiques). Il devra aussi comporter une gestion des appuis clavier et des clics souris.


\chapter{Analyse}


\section{}

\includegraphics{Textures/ok-button-2.png}

% \begin{lstlisting}
% Code
% \end{lstlisting}


% Annexes

\appendix
\addtolength{\textheight}{60mm}
\part*{Annexes}
\addtolength{\topmargin}{-50mm}
\definecolor{gray75}{gray}{0.75}
\newcommand{\hsp}{\hspace{20pt}}
\titleformat{\chapter}[block]{\Huge\bfseries}{\thechapter\hsp\textcolor{gray75}{|}\hsp}{0pt}{\Huge\bfseries}[\vskip -2em]

% \chapter{Annexe 1}

\end{document}

